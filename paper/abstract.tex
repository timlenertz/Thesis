\newenvironment{abstract2}{
  \vspace*{\fill}
  \begin{center}%
    \bfseries\abstractname
  \end{center}}%
  {\vfill}


\selectlanguage{english}
\begin{abstract2}
A review of different point cloud registration techniques is presented, and in particular the ICP (Iterative Closest Point) algorithm, and variations of it, are studied in greater detail. Experiments are conducted to tests its robustness for aligning point clouds of different resolutions. Based on the results, an attempt is made to develop a novel registration error metric which takes the arrangement of points on object surfaces into account in order to infer more information about the object's geometry. Some promising results were obtained.
\end{abstract2}

\selectlanguage{french}
\begin{abstract2}
Un état de l'art des différents algorithmes de recalage de nuages de points est présenté, et surtout l'algorithme ICP (itératif point le plus proche) et ses variations sont étudiés plus en détail. Des expériences sont accomplis dans le but de déterminer sa capacité de bien aligner des nuages de points ayant des résolutions très différentes. Sur base de ces résultats, une tentative est fait de développer une nouvelle métrique d'erreur de recalage qui prend en considération l'arrangement des points sur les surfaces des objets, afin d'inférer plus d'information sur leur géométrie. Quelques résultats prometteurs ont été obtenus.
\end{abstract2}
% It's still not a language

\clearpage