\chapter{Introduction}
In the context of 3D documentation projects, objects or environments are often digitized in the form of a \emph{point cloud}. This data recorded by 3D scanners consists of a set of point coordinates lying on the object surfaces. No information about the connectivity of the surface points is retained.

To get a full 3D model of an object, scans from different view points are recorded. Before merging the point clouds, they need to be put into exactly the same coordinate system. This is called \emph{registration}. Algorithms exists which automate this step, the most well-known of which is \gls{icp}.

For large scale 3D documentation projects, additional difficulties appear, like for example the registration of low resolution point clouds of an entire building, with high resolution point clouds focussed on a stone sculpture on the building's facade. To address this problem, an attempt is made in this thesis to develop an registration algorithm which looks at the arrangement of points in the lower resolution point cloud, to infer additional information about the object's geometry.

In chapter 2, some preliminary theory about 3D scanning and the mathematical notions on which registration algorithms are based, will be presented. Chapter 3 is a review of literature about existing registration algorithms, and in particular variants of \gls{icp}. Chapter 4 consists of two main parts: First some experiments with \gls{icp} registration are run to investigate its stability when the two point clouds are of different resolutions. Secondly, an registration error metric is developed which, as described, is based on the arrangement of points. Finally, chapter 5 describes the \cpp{} implementation of a software framework developed for this project.